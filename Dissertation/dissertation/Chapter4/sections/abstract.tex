

Protecting sensitive information against data exploiting attacks is an emerging research area in data mining. Over the past, several different methods have been introduced to protect individual privacy from such attacks while maximizing data-utility of the application. However, these existing techniques are not sufficient to effectively protect data owner privacy, especially in the scenarios that utilize visualizable data (e.g. images, videos) or the applications that require heavy computations for implementation. To address these problems, we propose a new dimension reduction-based method for privacy preservation. Our method generates dimension-reduced data for performing machine learning tasks and prevents a strong adversary from reconstructing the original data. We first introduce a theoretical approach to evaluate dimension reduction-based privacy preserving mechanisms, then propose a non-linear dimension reduction framework motivated by state-of-the-art neural network structures for privacy preservation. We conducted experiments over three different face image datasets (AT\&T, YaleB, and CelebA), and the results show that when the number of dimensions is reduced to seven, we can achieve the accuracies of 79\%, 80\%, and 73\% respectively and the reconstructed images are not recognizable to naked human eyes.